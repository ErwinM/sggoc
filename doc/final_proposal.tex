\documentclass[11pt]{article}
\usepackage[margin=0.5in]{geometry}
\usepackage{titling}
\usepackage[compact]{titlesec}

\setlength{\droptitle}{-5em}
\addtolength{\droptitle}{-5pt}

\title{Final Project Proposal}
\author{ Max Thrun | Samir Silbak }

\begin{document}
\maketitle

\section*{Project Title}
Sega Game Gear on a Chip (SGGoC)

\section*{Problem Statement}
As computer systems age it is often impossible to keep physically maintaining them.
For many old systems you cannot buy replacement parts and finding people who have the skill
to repair the original hardware is difficult. Recently, with the advent of Field Programmable
Gate Arrays (FPGAs) it is now possible to completely recreate all the original hardware and reimplement
everything in a hardware description language (HDL). The advantages of doing this is that you can
replace a huge multi chip solution with a single chip implementation. Doing so also helps future-proof
the system as the HDL is portable and standardized. Our project is a case study in the process of
reimplementing the Sega Game Gear gaming console which can be viewed as a specialized computer system. 
From our project we hope to educate and provide people with a case study on the process and implementation 
strategies that go into recreating these old computer systems.

\section*{Project Requirements}
The Game Gear hardware requirements can be easily broken down into submodules. Major components include the Zilog Z80 CPU, 
the Video Display Processor (VDP) which is a modified Texas Instruments TMS9918, the Sega IO controller, and the
game cartridge memory mappers. All of these components are considered archaic and are extremely hard to source which makes
the Game Gear an ideal system for this project. The reimplementation of the Zilog Z80 is outside the scope of this project and as such
we will be using the popular open source TV80 CPU. A memory management unit will need to be developed to coordinate the 
addressing of system RAM and the cartridge ROM. The cartridge ROM will need to be initially preloaded on the flash memory chip on
our development board. If time allows a proper bootloader may be developed to allow game ROMs to be selected off a
SD card. We plan on implementing the submodules in this order of priority: TV80 CPU, MMU, Sega Cartridge Memory Mapper, VDP,
Sega IO Controller, and Audio (YM2413). The metric used to determine if our design satisfies these requirements will simply
by the accuracy at which they reimplement the original functionality.

\section*{Functional Description}
Our final implementation will be a fully functioning Sega Game Gear running on an Altera DE-1 FPGA
development board. Video output will be via VGA to a computer monitor and input will be through some type of retro
gaming controller, such as the Sega Genesis controllers. Any Sega Game Gear ROM which uses the Sega mapper 
(we do not plan on implementing less common mappers) will be playable. Functional diagrams showing the interaction of
the submodules is shown on the next page.

\section*{Team Participants}
{\bf Max Thrun} - FPGAs / computer architecture / programming  
\\
{\bf Samir Silbak} - Linux / embedded systems / software development

\section*{Advisor}
Carla Purdy (carla.purdy@uc.edu)

\end{document}
